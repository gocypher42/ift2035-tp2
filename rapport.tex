\documentclass[12pt, letterpaper]{article}
\usepackage[utf8]{inputenc}

\setlength{\parskip}{0.5em}

\title{IFT 2035 \\ Travail pratique 2 - Interpréteur H2035 \\ Rapport }
\author{ Marie-Anne Prud'Homme-Maurice (1054064) 
\\ Olivier Guénette (20154866)}
\date{13 décembre 2021}

\begin{document}
\maketitle

\section*{Tentons d'être logique} 

Dans le cadre du cours IFT 2035. Il nous a été demandé de concevoir un 
elaborateur et un evaluateur du language h2034 en utilisant le language 
de programmation logique Prolog. Le travail a pour but d'implanter une 
fonction qui décompose les lignes de code pour faire une inférence des types
ainsi que de remplacer les appels de variable pour utiliser les indices de 
Brujn. Par la suite, un autre fonction eval permet d'effectuer les opérations 
appropirées.

Ce rapport décrit notre processus d'analyse, les problèmes rencontrés et nos
solutions.

\section*{Apprentissage de Prolog}

\section*{Utilisation des indice de Bruijn}

\section*{Détermination des ? par inférence de type}

\section*{Implentation de let mutuellement recursif}

\section*{Association de valeur aux variables}

\section*{Conclusion}

\end{document}
