\documentclass[12pt, letterpaper]{article}
\usepackage[utf8]{inputenc}

\setlength{\parskip}{0.5em}

\title{IFT 2035 \\ Travail pratique 2 - Interpréteur H2035 \\ Rapport }
\author{ Marie-Anne Prud'Homme-Maurice (1054064) 
\\ Olivier Guénette (20154866)}
\date{13 décembre 2021}

\begin{document}
\maketitle

\section*{Tentons d'être logique} 

Dans le cadre du cours IFT 2035. Il nous a été demandé de concevoir un 
elaborateur et un evaluateur du language h2034 en utilisant le language 
de programmation logique Prolog. Le travail a pour but d'implanter une 
fonction qui décompose les lignes de code pour faire une inférence des types
ainsi que de remplacer les appels de variable pour utiliser les indices de 
Brujn. Par la suite, un autre fonction eval permet d'effectuer les opérations 
appropirées.

Ce rapport décrit notre processus d'analyse, les problèmes rencontrés et nos
solutions.

\section*{Apprentissage de Prolog}

Comme une majorité de personnes dans la classe, cet exercice fût notre premier
vrai travail pratique en programmation logique. Pour débuter le projet, nous 
avons passer un peu de temps à lire et faire des recherches sur le language 
pour venir palier à notre manque de connaissance.\\

Voici quelques ressources utilisées: 
\begin{itemize}
    \item https://www.tutorialspoint.com/prolog/index.htm
    \item https://ocw.upj.ac.id/files/Textbook-TIF212-Prolog-Tutorial-3.pdf
    \item https://www.geeksforgeeks.org/prolog-an-introduction/
\end{itemize}

\section*{Utilisation des indice de Bruijn}

L'utilisation des indices de Bruijn est un concept interessant pour l'implifier
la référence au valeur des variables dans un environnment donné. Cependant, 
nous avions de la difficulté à faire les liens entre les valeurs des indices 
dans les exemples donnée. Notre intuition pensait que les indices devait 
posséder d'autres valeur que celles mentionnées.

Pour résoudre cette situation, nous avons faire des sous-routine de test 
pour nous donner plus de vision sur le contenue de l'environnment en imprimant
sont contenue au moment de trouver l'indice de la variable.

De plus, lors de l'évaluation des fonctions, des erreurs sont survenue ce qui
nous a induit vers un processus de recherche pour corriger le problème de 
référence.

\section*{Association de valeur aux variables indéfinies}

Dans prolog, des variables variables peuvent être déclarer, mais instancier 
vraiment plus tard dans l'exécution du programme. Ceci étati plus une surprise 
de notre pars, car aillant plus travailler avec Java, Python et Javascript
il nous était pas intuitif de passer des variables non définie dans des fonctions
comme paramètres. 

Un exemple concrete, serait l'élaboration des équation contenant un lambda.
Dans le code fournie, la variable du lambda est insérée au début de l'environnment
avec une référence vers une variable de type inconnue. Cette variable est ensuite 
utiliser pour créé le type du lamabda. Cette utilisation de variable était un
concepte différent pour nous.

La solution a été de faire plus de petit exercices en utilisant les ressources 
mentionnées plus haut. Ainsi que de trouver une facons d'associer des valeurs 
à ces variables.

\section*{Détermination des ? par inférence de type}

\section*{Implentation de let mutuellement recursif}

\section*{Conclusion}

\end{document}
